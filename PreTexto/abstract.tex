%%%% ABSTRACT
%%
%% Versão do resumo para idioma de divulgação internacional.

\begin{abstractutfpr}%% Ambiente abstractutfpr
This undergraduate thesis explores the feasibility and development of a low-cost multimeter prototype. In the laboratories of the Federal Technological University of Paraná, tests are conducted with the aim of providing practical learning to students in the Electrical Engineering program and related fields. For the execution of these experiments, the use of metrology equipment is essential, along with the circuits that will be analyzed.
In this context, the development of an oscilloscope multimeter with communication via WiFi is proposed, capable of performing measurements in various experiments conducted in the laboratories. The project was designed to be simple and educational, allowing the integration of the human-machine interface through the screen of a smartphone or computer connected to the device, thus enabling the visualization of waveforms and the acquisition of measurements relevant to the projects.
The prototype consists of hardware, software, and firmware, developed entirely using open source platforms. Furthermore, all its components have been thoroughly discretized, aiming to maximize the replicability of the project.
It was possible to validate the prototype regarding its current and voltage scales as proposed, with results shown on browser, being possible to measure them simultaniously with very close results to conventional equipment.
The concept of modularity was also explored in the system's structure, aiming to reduce the complexity, facilitate assembly and maintenance of the equipment, but it was not implemented. That being said, future development is necessary to make it possible to measure more than a single phase utilizing replicas of the device.
\end{abstractutfpr}