%%%% ABSTRACT
%%
%% Versão do resumo para idioma de divulgação internacional.

\begin{abstractutfpr}%% Ambiente abstractutfpr
This thesis explores the feasibility and development of a low-cost multimeter prototype. In the laboratories of the Federal Technological University of Paraná, tests are conducted with the aim of providing practical learning to students in the Electrical Engineering program and related fields. For the execution of these experiments, the use of metrology equipment is essential, along with the circuits that will be analyzed.
In this context, the development of an oscilloscope multimeter with communication via WiFi is proposed, capable of performing measurements in various experiments conducted in the laboratories. The project was designed to be simple and educational, allowing the integration of the human-machine interface through the screen of a smartphone or computer connected to the device, thus enabling the visualization of waveforms and the acquisition of measurements relevant to the projects.
The prototype consists of Hardware, Software, and Firmware, developed entirely using Open-Source platforms. Furthermore, all its components have been thoroughly discretized, aiming to maximize the replicability of the project.
It was possible to develop and validate the prototype within the proposed voltage and current scales; however, due to difficulties with development and complexity, it was not possible to implement modularity. Thus, future development is necessary to enable the reading of more than one phase using replicas of the device.
\end{abstractutfpr}