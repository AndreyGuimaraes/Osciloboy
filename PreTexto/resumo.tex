%%%% RESUMO
%%
%% Apresentação concisa dos pontos relevantes de um texto, fornecendo uma visão rápida e clara do conteúdo e das conclusões do
%% trabalho.


\begin{resumoutfpr}%% Ambiente resumoutfpr
  Esta monografia explora a viabilidade e desenvolvimento do protótipo de um multimedidor de baixo custo. Nos laboratórios da Universidade Tecnológica Federal do Paraná, são realizados testes com a finalidade de aprendizado dos alunos do curso de Engenharia Elétrica e demais. Para tanto, são necessários equipamentos de metrologia, além dos circuitos a serem analizados. Para tanto, propõe-se um multimedidor osciloscópio com comunicação \textit{WiFi} para fazer medidas dos vários experimentos promovidos nos laboratórios. Visando ser o mais simples e didático possível, a integração da interface homem-máquina se dá pela tela de um \textit{smartphone} conectado ao dispositivo, tornando assim possível a visualização de ondas, além das medidas relevantes aos projetos. Este protótipo, composto por \textit{Hardware}, \textit{Software} e \textit{Firmware} é desenvolvido em sua totalidade utilizando-se plataformas \textit{Open-Source} disponibilizadas e também tem todos os seus materiais discretizados, visando-se a maior replicabilidade possível. ****testes e conclusão****
\end{resumoutfpr}
