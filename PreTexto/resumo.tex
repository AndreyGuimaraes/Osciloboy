%%%% RESUMO
%%
%% Apresentação concisa dos pontos relevantes de um texto, fornecendo uma visão rápida e clara do conteúdo e das conclusões do
%% trabalho.


\begin{resumoutfpr}%% Ambiente resumoutfpr
Esta monografia explora a viabilidade e o desenvolvimento de um protótipo de multimedidor de baixo custo. Nos laboratórios da Universidade Tecnológica Federal do Paraná, são realizados testes com o objetivo de proporcionar aprendizado prático aos alunos do curso de Engenharia Elétrica e áreas afins. Para a realização desses experimentos, é essencial o uso de equipamentos de metrologia, além dos circuitos que serão analisados.
Neste contexto, propõe-se o desenvolvimento de um multimedidor osciloscópio com comunicação via \textit{WiFi}, capaz de realizar medições em diversos experimentos realizados nos laboratórios. O projeto foi concebido para ser simples e didático, permitindo a integração da interface homem-máquina por meio da tela de um \textit{smartphone} ou computador conectado ao dispositivo. Desta maneira, é possível a visualização de formas de onda e a obtenção das medições relevantes para os projetos.
O protótipo é composto por \textit{hardware}, \textit{software} e \textit{firmware}, desenvolvidos integralmente com o uso de plataformas \textit{open source}. Além disso, todos os seus componentes foram detalhadamente discretizados, com o intuito de maximizar a replicabilidade do projeto.
Foi possível realizar o desenvolvimento e validação do protótipo dentro das escalas de tensão e corrente propostas, com resultados apresentados em \textit{browser}, sendo estas possíveis de ser medidas simultâneamente com resultados próximos a equipamentos convencionais.  
Também foi explorado o conceito de modularidade na representação da estrutura do sistema, visando diminuir sua complexidade e facilitar a montagem e manutenção do equipamento, porém esta finalmente não foi implementada. Dessa forma, se torna necessário desenvolvimento futuro para que seja possível realizar a leitura de mais de uma fase utilizando réplicas do dispositivo.
\end{resumoutfpr}
