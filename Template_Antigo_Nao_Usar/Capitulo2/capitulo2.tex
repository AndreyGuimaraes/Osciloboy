%%%% CAPÍTULO 2 - REVISÃO DA LITERATURA (OU REVISÃO BIBLIOGRÁFICA, ESTADO DA ARTE, ESTADO DO CONHECIMENTO)
%%
%% O autor deve registrar seu conhecimento sobre a
%% literatura básica do assunto, discutindo e 
%% comentando a informação já publicada. A revisão deve
%% ser apresentada, preferencialmente, em ordem
%% cronológica e por blocos de assunto, procurando
%% mostrar a evolução do tema.

%% Título e rótulo de capítulo (rótulos não devem conter caracteres especiais, acentuados ou cedilha)
\chapter{Revisão da Literatura}\label{cap:revisaodaliteratura}

\section{Aquisição de Sinal}\label{sec:aqsinal}
A aquisição de sinais de tensão e corrente são diferentes entre si, bem como existem diferenças caso a medição seja monofásica, trifásica, simultânea ou assíncrona, em um ou mais canais.

\subsection{Aquisição monocanal}\label{subsec:monocanal}

Verificar \ref{sec:acronimos}

Também é possível ver em \ref{tab:tab4}