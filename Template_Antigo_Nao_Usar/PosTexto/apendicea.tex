%%%% APÊNDICE A
%%
%% Texto ou documento elaborado pelo autor, a fim de complementar sua argumentação, sem prejuízo da unidade nuclear do trabalho.

%% Título e rótulo de apêndice (rótulos não devem conter caracteres especiais, acentuados ou cedilha)
\chapter{Título do Apêndice A com um Texto Muito Longo que Pode Ocupar Mais de uma Linha}\label{cap:apendicea}

Quando houver necessidade pode-se apresentar como apêndice documento(s) auxiliar(es) e/ou complementar(es) como: legislação, estatutos, gráficos, tabelas, etc. Os apêndices são enumerados com letras maiúsculas: \autoref{cap:apendicea}, \autoref{cap:apendiceb}, etc.

No \latex\ apêndices são editados como capítulos. O comando \verb|\appendix| faz com que todos os capítulos seguintes sejam considerados apêndices.

Apêndices complementam o texto principal da tese com informações para leitores com especial interesse no tema, devendo ser considerados leitura opcional, ou seja, o entendimento do texto principal da tese não deve exigir a leitura atenta dos apêndices.

Apêndices usualmente contemplam provas de teoremas, deduções de fórmulas matemáticas, diagramas esquemáticos, gráficos e trechos de código. Quanto a este último, código extenso não deve fazer parte da tese, mesmo como apêndice. O ideal é disponibilizar o código na Internet para os interessados em examiná-lo ou utilizá-lo.

%% Título e rótulo de seção (rótulos não devem conter caracteres especiais, acentuados ou cedilha)
\section{Título da Seção Secundária do Apêndice A}\label{sec:secaoapendicea}

Exemplo de seção secundária em apêndice (\autoref{sec:secaoapendicea} do \autoref{cap:apendicea}).

%% Título e rótulo de seção (rótulos não devem conter caracteres especiais, acentuados ou cedilha)
\subsection{Título da Seção Terciária do Apêndice A}\label{subsec:subsecaoapendicea}

Exemplo de seção terciária em apêndice (\autoref{subsec:subsecaoapendicea} do \autoref{cap:apendicea}).

%% Título e rótulo de seção (rótulos não devem conter caracteres especiais, acentuados ou cedilha)
\subsubsection{Título da seção quaternária do Apêndice A}\label{subsubsec:subsubsecaoapendicea}

Exemplo de seção quaternária em apêndice (\autoref{subsubsec:subsubsecaoapendicea} do \autoref{cap:apendicea}).

%% Título e rótulo de seção (rótulos não devem conter caracteres especiais, acentuados ou cedilha)
\paragraph{Título da seção quinária do Apêndice A}\label{para:paragraphapendicea}

Exemplo de seção quinária em apêndice (\autoref{para:paragraphapendicea} do \autoref{cap:apendicea}).
