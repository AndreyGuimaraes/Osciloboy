%%%% ABSTRACT
%%
%% Versão do resumo para idioma de divulgação internacional.

\begin{abstractutfpr}%% Ambiente abstractutfpr
The abstract should be drafted in the third person singular with the verb in the active voice, not exceeding one page (from 150 to 500 words, according to ABNT NBR 6028), avoiding the use of paragraphs in the middle of the summary, formulas, equations and symbols. Start the abstract setting the work in the general context, presenting the objectives, describe the methodology adopted, reporting the contribution itself, commenting on the results and finally present the conclusions of the most important work. The keywords should appear below the abstract, preceded by the expression Keywords. To define the keywords (and their corresponding portuguese in the \textit{resumo}) query in Authorities Catalog Topic term of the National Library, available at: \url{http://acervo.bn.br/sophia_web/index.html}.
\end{abstractutfpr}
