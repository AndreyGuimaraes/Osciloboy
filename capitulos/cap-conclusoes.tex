%%%% CAPÍTULO 5 - CONCLUSÕES E PERSPECTIVAS
%%
\chapter{Conclusão}\label{cap:conclusoeseperspectivas}

A finalidade principal deste trabalho foi desenvolver um multimedidor com comunicação \textit{wireless} de baixo custo, capaz de medir tensão e corrente simultaneamente. De maneira sucinta, pode-se concluir que a produção do protótipo e a prova de conceito sobre a sua funcionalidade trouxeram resultados promissores para subsequentes pesquisas.

As especificações do \gls{ADC} e do microcontrolador são as de maior importância para o desenvolvimento de um protótipo com esta finalidade. Isto se deve às necessidades a serem atendidas ou o propósito do instrumento de medição a ser elaborado, como precisão e ranges de leitura, quais tipos de leitura serão feitas, a interface homem-máquina, velocidade de transmissão de dados, entre várias outras. Uma consideração de suma importância também é a compatibilidade entre estes componentes.

Para alcançar o objetivo de leitura paralela de tensão e corrente, foi necessário o desenvolvimento de um circuito isolado internamente, garantindo a integridade dos sinais e dos componentes do circuito. Isto se apresentou como um grande desafio quando se trata de baixo custo e a replicabilidade do protótipo desenvolvido, pois soluções já existentes são em grande maioria custosas e difíceis de se encontrar em mercados não especializados.

Este trabalho demonstra o desenvolvimento de um \textit{firmware} que é necessário para o funcionamento do equipamento como um todo. Este integra todo o resultado da pesquisa e possibilita a comunicação entre o ADC e a plataforma de tratamento de dados. Esta pode ser acessada por dispositivos externos como smartphones ou computadores, assim como deixa estruturada uma futura modularização, tornando este uma das partes mais críticas do projeto, assim como desafiadora.

É interessante pontuar que as leituras true RMS de multímetros comercias não necessariamente conseguiriam fazer a leitura corretamente quando se tem um circuito que gera sinais com componentes DC devido ao acoplamento AC utilizado por estes multímetros para realizar sua leitura. O protótipo neste trabalho desenvolvido não possui acoplamento AC e considera a onda pura para o cálculo, resultando em um valor mais condizente com a realidade.

Para fazer esta medida com um multímetro true RMS seria necessário fazer uma medida AC, uma DC e calcular o resultado final com a equação $RMS=\sqrt{CArms^2+CC^2}$". Multímetros que dispõe tal utilidade são custosos monetariamente, o que é uma vantagem para o desenvolvimento do protótipo neste trabalho proposto. 

Considerando as especificações adotadas para o protótipo, as tabelas \ref{tab:resultados-precis} e \ref{tab:resultados-func} demonstram os resultados obtidos, considerando todos os aspectos da pesquisa.

\begin{table}[!h]
    \centering
    \caption{Precisões solicitadas e resultados}
    \vspace*{5mm}
    \label{tab:resultados-precis}
    \begin{tabular}{ l l l l }
        \hline
        \textbf{Parâmetro} & \textbf{Precisão Solicitada} & \textbf{Precisão Atingida} & \textbf{Status}         \\ \hline
        Tensão (V)         & < 2\%                        & entre 0,47\% e 1,46\%      & Dentro de Especificação \\ 
        Tensão (mV)        & < 2\%                        & entre 31,47\% e 37,23\%    & Fora de Especificação   \\ 
        Corrente (A)       & < 5\%                        & 0,047\%                    & Dentro de Especificação \\ 
        Corrente (mA)      & < 5\%                        & 2,72\%                     & Dentro de Especificação  \\ 
        Corrente ($\mu$A)  & < 5\%                        & 2,51\%                     & Dentro de Especificação   \\ \hline
    \end{tabular}
    \fonte{}
\end{table}

\begin{table}[!h]
    \centering
    \caption{Implementação de funções de acordo com especificações}
    \vspace*{5mm}
    \label{tab:resultados-func}
    \begin{tabular}{ l l }
        \hline
        \textbf{Função}                                  & \textbf{Status} \\ \hline
        Leitura Simultânea de Tensão e Corrente True RMS & Implementado  \\ 
        Exibição de Formas de Onda de Tensão e Corrente  & Implementado  \\ 
        Cálculo de Potências Ativa, Reativa e Aparente   & Implementado  \\ 
        Cálculo do Fator de Potência                     & Implementado  \\ \hline
        Modularização e Leitura de Multiplas Fases       & Não Implementado  \\ \hline
    \end{tabular}
    \fonte{}
\end{table}

De acordo com as especificações de aspectos construtivos, foi adicionado um borne a mais para a corrente visto a necessidade de se trocar a entrada desta para leituras de A, mA e $\mu$A. A utilização de fontes internas isoladas foi atendida e o tipo de interface homem-máquina também foi atendido, se dando por smartphone ou computador.

Nota-se que o dispositivo também oferece a leitura da frequência do sinal a ser medido. Além disto, de acordo com o teste utilizando-se o circuito de \textit{Wheatstone}, a leitura simultânea de sinais de corrente e tensão em partes diferentes do circuito sem interferências é provada.

Quanto aos resultados dos testes, nota-se que as especificações foram em sua grande maioria atendidas, exceto o range de mV, devido a uma má calibração de seus parâmetros. Este pode ser calibrado posteriormente por meio do firmware integrado ao dispositivo para o aperfeiçoamento da aquisição destes sinais e então sua leitura.

A funcionalidade de modularização não foi implementada nesta etapa devido à sua complexidade de implementação, por limitações de protocolos de comunicação \textit{wireless} atrelados à plataforma e o hardware escolhidos e também por limitações de tempo. 

Com isso, propõe-se para trabalhos futuros a implementação da modularidade do sistema e com isso a expansão e otimização do software e firmware.

Finalmente, evidenciam-se alguns dos diferenciais do protótipo projetado:

\begin{itemize}
    \item True RMS;
    \item Leitura diferencial isolada para redução de ruídos, o que previne problemas;
    \item Apresenta forma de ondas de tensão e corrente simultaneamente, junto de potência (W, VAr, VA, FP) e frequência;
    \item Possibilidade de múltiplos dispositivos conectados;
    \item Plataforma aberta para futuro desenvolvimento;
    \item Custo adequado à finalidade.
\end{itemize}