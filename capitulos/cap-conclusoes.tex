%%%% CAPÍTULO 5 - CONCLUSÕES E PERSPECTIVAS
%%
\chapter{Conclusão}\label{cap:conclusoeseperspectivas}

O propósito deste trabalho foi o desenvolvimento de um multimedidor de baixo custo modular capaz de medir tensão e corrente paralelamente. Devido a complexidade da modularização, essa parte não foi implementada, porém, foi analisada e encaminhada para a funcionalidade em trabalhos futuros.

Para o desenvolvimento do \textit{hardware}, pesquisou-se em trabalhos científicos, manuais, \textit{datasheets}, livros, blogs, entre outras fontes para se ter um melhor entendimento de como funciona um circuito de leitura de sinais elétrico. Para a proteção deste circuito, muito foi pesquisado sobre proteção de entrada para tais instrumentos. Também para alcançar o objetivo de leitura paralela de tensão e corrente, foi necessário o desenvolvimento de um circuito isolado internamente, o que se apresentou como um grande obstáculo.

Foi colocado em questão o funcionamento do ADC, o componente mais importante para o objetivo e que também se pôs como a maior barreira para o desenvolvimento deste protótipo. Junto com este, o microcontrolador escolhido também foi estudado sumariamente para se desenvolver o \textit{firmware} necessário para o funcionamento do pacote como um todo. Este \textit{firmware} integra todo o resultado da pesquisa e assim possibilita a comunicação entre o ADC e a plataforma de tratamento de dados.

Foi possível se estabelecer a medição e comunicação dos dados com um dispositivo externo, bem como alcançar a precisão necessária para se observar a forma de onda e obter medições adequadas para o ensino.