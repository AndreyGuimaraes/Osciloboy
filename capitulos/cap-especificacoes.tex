\chapter{Especificações}\label{cap:especificacoes}
Baseado em relatos dos professores das disciplinas de Eletricidade e Magnetismo, Circuitos A e Circuitos B da \gls{utfpr} de Curitiba. Além disso, dados os equipamentos utilizados nos laboratórios dessas disciplinas, bem como os utilizados no \gls{SEMAP}(Setor de Almoxarifado/Manutenção dos Laboratórios) da universidade.

Definiram-se as seguintes especificações elétricas para o multímetro desenvolvido, conforme a \autoref{tab:specstable}:

\begin{table}[!ht]
    \centering
    \caption{Resolução e precisão necessárias para o dispositivo}
    \label{tab:specstable}
    \begin{tabular}{|l|l|l|}
    \hline
        \textbf{Especificação} & \textbf{Tensão} & \textbf{Corrente} \\ \hline
        \textbf{Faixa de Leitura} & 0 - 220V & 0 - 10 A \\ \hline
        \textbf{Resolução} & 3 1/2 dígitos & 3 1/2 dígitos \\ \hline
        \textbf{Precisão} & <2\% & <5\% \\ \hline
    \end{tabular}   
\end{table}

Foram também definidas as especificações quanto a construção e sistemas do dispositivo, conforme segue:

\begin{itemize}
    \item Número de bornes para tensão: 2
    \item Número de bornes para corrente: 2
    \item Tipo de alimentação: Fonte interna isolada
    \item Display de dados: Através de Wifi para dispositivos com acesso a navegador web
\end{itemize}

Os dados que o dispositivo será capaz de apresentar ao usuário são:

\begin{itemize}
    \item Formas de onda de tensão e corrente simultâneas,
    \item Tensão e corrente RMS,
    \item Potência ativa, reativa, aparente,
    \item Fator de potência.
\end{itemize}

O multímetro será fabricado em placa de circuito impressa e montado em um módulo de bancada conforme os que os laboratórios da \gls{utfpr} possuem.