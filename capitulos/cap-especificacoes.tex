\chapter{Especificações e Premissas Adotadas}\label{cap:especificacoes}

Neste capítulo, serão apresentadas tanto as especificações do protótipo quanto as premissas adotadas, tendo em vista os objetivos do projeto.

\section{Especificações}\label{spec}
Baseado em relatos dos professores das disciplinas de Fenômenos Eletromagnéticos e Análise de Circuitos A e B da \gls{utfpr} de Curitiba, os equipamentos utilizados nos laboratórios destas disciplinas, bem como os utilizados no \gls{SEMAP}(Setor de Almoxarifado/Manutenção dos Laboratórios) da universidade, definiram-se as seguintes especificações elétricas para o multímetro desenvolvido, conforme a \autoref{tab:specstable}:

\begin{table}[!ht]
    \centering
    \caption{Resolução e precisão necessárias para o dispositivo}
    \label{tab:specstable}
    \begin{tabular}{|l|l|l|}
        \hline
        \textbf{Especificação}    & \textbf{Tensão} & \textbf{Corrente} \\ \hline
        \textbf{Faixa de Leitura} & 0 - 220V        & 0 - 10 A          \\ \hline
        \textbf{Precisão}         & <2\%            & <5\%              \\ \hline
    \end{tabular}
    \fonte{}
\end{table}

Foram também definidas as especificações quanto a construção e sistemas do dispositivo, conforme segue:

\begin{itemize}
    \item Número de bornes para tensão: 2
    \item Número de bornes para corrente: 2
    \item Tipo de alimentação: Fonte interna isolada
    \item Display de dados: \textit{Smartphones} com acesso a navegador, por \textit{WiFi}.
\end{itemize}

Os dados que o dispositivo será capaz de apresentar ao usuário são:

\begin{itemize}
    \item Formas de onda de tensão e corrente simultâneas,
    \item Tensão e corrente RMS,
    \item Potência ativa, reativa, aparente,
    \item Fator de potência.
\end{itemize}

\section{Premissas Adotadas}\label{premissas1}

As premissas deste trabalho são de suma importância, visto que o objetivo do mesmo é projetar um protótipo com um alto nível de replicabilidade e sendo o mais barato possível. Para isso, serão utilizadas as plataformas gratúitas descritas em seguida e também será fornecido um link para o repositório no qual será desenvolvido o software.

\subsection{Hardware}\label{proto}

Primeiramente, nota-se que o circuito e a \textit{\gls{PCB}} (\textit{Printed Circuit Board}) estão sendo desenvolvidos em uma plataforma chamada easyEDA. Esta plataforma, além de fornecer todo um sistema para simulações e desenvolvimento, possúi uma \textit{supply chain} integrada, tornando extremamente simplificado o desenvolvimento e a prototipagem do circuito, sendo possível escolher já as footprints de todos os componentes e também já verificar a disponibilidade destes no mercado.

O roteamento das trilhas de cobre, definição de sua espessura e também o a modelagem em 3D da \textit{\gls{PCB}} são disponíveis nesta plataforma, tornando-a extremamente versátil. Tudo isto é fornecido de forma gratuita pelo site.

Assim, além de ser desenvolvido em uma plataforma gratúita, o projeto desenvolvido será disponibilizado para acesso pelo link disponibilizado no final deste trabalho..

\subsection{Software e Firmware}\label{softw}

O desenvolvimento completo do \textit{software} utilizado neste projeto se dá pelo editor de código chamado \textit{Visual Studio Code}, ou em abreviação, \gls{VSCode}. Esta plataforma é gratúita e oferece suporte para todas as linguagens de programação.

Dentro deste editor, existem 3 vetores de programação que serão a base do \textit{software} e \textit{firmware}. Primeiramente, se é utilizado \gls{HTML} 5 e \gls{JS} (JavaScript) para a construção do aplicativo web que servirá de monitor para os dados obtidos pelos sensores.

Para o código em Arduino que controlará o ESP32 e também o \textit{Firmware}, será utilizado o PlatformIO, uma \textit{\gls{IDE}} (\textit{Integrated Development Environment}) gratúita que é uma extensão do VSCode.

Por último, será utilizado o \gls{Git}, que é um software de controle e versionamento de código, tornando assim possível a disponibilização de todo o software desenvolvido neste projeto e também seu versionamento por meio de um site chamado \gls{GitHub}. Tal software também pode ser utilizado como uma extensão do VSCode, aumentando e simplificando ainda mais a disponibilidade do software desenvolvido. O link para o repositório está disponibilizado no final deste trabalho.



