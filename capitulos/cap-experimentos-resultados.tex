%%%% CAPÍTULO 4 - RESULTADOS E DISCUSSÃO

\chapter{Testes e Resultados}\label{cap:resultados}

Os testes feitos para a prova de conceito e também a caracterização da classe do protótipo foram as faixas de leitura, resolução e também a comparação das leituras obtidas (tanto \gls{CA} quanto \gls{CC}) do protótipo, de um multímetro e de um osciloscópio.

Primeiramente, são apresentadas as faixas de leitura e sua resolução desejada para leituras de tensão e corrente, pela tabelas \ref{tab:faixasCA} e \ref{tab:faixasCC}.


\begin{table}[!ht]
    \centering
    \caption{Faixas \gls{CA}}
    \label{tab:faixasCA}
    \begin{tabular}{|l|l|l|}
        \hline
        \textbf{Faixa} & \textbf{Resolução} & \textbf{\gls{PGA}} \\ \hline
        220 V          & (valor)            & (valor)            \\ \hline
        20 V           & (valor)            & (valor)            \\ \hline
        2 V            & (valor)            & (valor)            \\ \hline
        200 mV         & (valor)            & (valor)            \\ \hline
        10 A           & (valor)            & (valor)            \\ \hline
        1 A            & (valor)            & (valor)            \\ \hline
        10 mA          & (valor)            & (valor)            \\ \hline
    \end{tabular}
\end{table}

\begin{table}[!ht]
    \centering
    \caption{Faixas \gls{CC}}
    \label{tab:faixasCC}
    \begin{tabular}{|l|l|l|}
        \hline
        \textbf{Faixa} & \textbf{Resolução} & \textbf{\gls{PGA}} \\ \hline
        220 V          & (valor)            & (valor)            \\ \hline
        20 V           & (valor)            & (valor)            \\ \hline
        2 V            & (valor)            & (valor)            \\ \hline
        200 mV         & (valor)            & (valor)            \\ \hline
        10 A           & (valor)            & (valor)            \\ \hline
        1 A            & (valor)            & (valor)            \\ \hline
        10 mA          & (valor)            & (valor)            \\ \hline
    \end{tabular}
\end{table}

Nas seguintes tabelas, são comparadas as leituras de tensão entre o protótipo e um osciloscópio e também de corrente entre o protótipo e um multímetro.

\begin{table}[!ht]
    \centering
    \caption{Comparação leituras de tensão}
    \label{tab:compT}
    \begin{tabular}{|l|l|l|}
        \hline
        \textbf{Valor da fonte} & \textbf{Leitura Protótipo} & \textbf{Leitura Osciloscópio} \\ \hline
        220 V                   & (valor)                    & (valor)                       \\ \hline
        20 V                    & (valor)                    & (valor)                       \\ \hline
        2 V                     & (valor)                    & (valor)                       \\ \hline
        200 mV                  & (valor)                    & (valor)                       \\ \hline
    \end{tabular}
\end{table}

\begin{table}[!ht]
    \centering
    \caption{Comparação leituras de corrente}
    \label{tab:compA}
    \begin{tabular}{|l|l|l|}
        \hline
        \textbf{Valor da fonte} & \textbf{Leitura Protótipo} & \textbf{Leitura Multímetro} \\ \hline
        10 A                    & (valor)                    & (valor)                     \\ \hline
        1 A                     & (valor)                    & (valor)                     \\ \hline
        10 mA                   & (valor)                    & (valor)                     \\ \hline
    \end{tabular}
\end{table}

