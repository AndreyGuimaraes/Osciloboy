%%%% CAPÍTULO 3 - MATERIAL E MÉTODOS (PODE SER OUTRO TÍTULO DE ACORDO COM O TRABALHO REALIZADO)

\chapter{Materiais e Metodologia}\label{cap:materialemetodos}

Neste capítulo, será discutido e analizado todos os processos relacionados com o desenvolvimento do protótipo, a justificativa de cada escolha tomada, assim como um \textit{\gls{BOM}} (\textit{Book of Materials}) consolidado, com os preços na data do trabalho, para a confecção de um protótipo.


\section{Metodologia}\label{sec:metodo}

\subsection{Lógica de funcionamento}\label{logic}
\subsection{Processo e motivação}\label{proccess}
\subsection{ESP32}\label{ESPw}
\subsection{Cálculos}\label{calc}
\subsection{Circuito}\label{circuit}
\subsection{Software}\label{softwar}
\subsection{Testes}\label{testesw}


\section{Materiais}\label{sec:materiais}

\begin{table}[!ht]
    \centering
    \caption{\textit{Book of Materials}}
    \label{tab:specstable}
    \begin{tabular}{|l|l|l|}
        \hline
        \textbf{Material}                      & \textbf{Quantidade} & \textbf{Preço Unidade} \\ \hline
        \textbf{ESP32}                         & 01                  & N/A                    \\ \hline
        \textbf{ADC ADS1015}                   & 02                  & N/A                    \\ \hline
        \textbf{Buzzer THT}                    & 01                  & N/A                    \\ \hline
        \textbf{Capacitor 100 nF}              & 16                  & N/A                    \\ \hline
        \textbf{Capacitor 330 pF}              & 04                  & N/A                    \\ \hline
        \textbf{Capacitor 10 µF}               & 04                  & N/A                    \\ \hline
        \textbf{Capacitor 1 µF}                & 06                  & N/A                    \\ \hline
        \textbf{Diodo Schottky 1N5819}         & 04                  & N/A                    \\ \hline
        \textbf{Diodo 1N4007}                  & 10                  & N/A                    \\ \hline
        \textbf{Trimpot 1 K$\Omega$}           & 04                  & N/A                    \\ \hline
        \textbf{Transistor BC547}              & 01                  & N/A                    \\ \hline
        \textbf{Resistor 220 $\Omega$}         & 06                  & N/A                    \\ \hline
        \textbf{Resistor 1 k$\Omega$}          & 13                  & N/A                    \\ \hline
        \textbf{Resistor 330 $\Omega$}         & 02                  & N/A                    \\ \hline
        \textbf{Resistor 0 $\Omega$}           & 04                  & N/A                    \\ \hline
        \textbf{Resistor 3k3 $\Omega$}         & 01                  & N/A                    \\ \hline
        \textbf{Resistor 1\% 2 M$\Omega$}      & 04                  & N/A                    \\ \hline
        \textbf{Resistor 1\% 10 k$\Omega$}     & 04                  & N/A                    \\ \hline
        \textbf{Resistor 1\% 100 $\Omega$}     & 04                  & N/A                    \\ \hline
        \textbf{Resistor 1\% 1 $\Omega$}       & 04                  & N/A                    \\ \hline
        \textbf{Resistor SMD 1\% 10 m$\Omega$} & 01                  & N/A                    \\ \hline
        \textbf{Varistor S05K385}              & 02                  & N/A                    \\ \hline
        \textbf{Borne KRE2}                    & 03                  & N/A                    \\ \hline
        \textbf{Borne KRE3}                    & 02                  & N/A                    \\ \hline
        \textbf{Alavanca 2 posições}           & 01                  & N/A                    \\ \hline
        \textbf{LM317T}                        & 02                  & N/A                    \\ \hline
        \textbf{LM358}                         & 02                  & N/A                    \\ \hline
        \textbf{LM7805}                        & 02                  & N/A                    \\ \hline
        \textbf{6N137}                         & 06                  & N/A                    \\ \hline
        \textbf{Barra de Pinos Fêmea 40x1}     & 02                  & N/A                    \\ \hline
    \end{tabular}
\end{table}