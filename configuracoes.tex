%%%% configuracoes.tex, 2022/05/02, 2.4a

%% ####################################################
%%
%% >> Atenção - Leia isso antes de usar esse template<< 
%%
%% Esse template foi desenvolvido por professores,  com a intenção de ajudar os alunos com as entregas na biblioteca. Não há uma equipe especializada e dedicada mantendo tal template, mas sim professores trabalhando além das suas funções básicas, que são: ensino, pesquisa e extensão.
%
%% Também os mantenedores deste template não são especializados em LaTeX, muito menos em normas da ABNT. Todos que contribuíram com o template fizeram isso visando deixá-lo o mais próximo possível das normas da ABNT e das regras, anseios e expectativas da biblioteca da UTFPR. É muito importante entender que os desenvolvedores do template não têm relação direta com a biblioteca ou com a ABNT. Ou seja, não são os desenvolvedores do template que ditam as regras e normas dos textos que devem ser entregues à biblioteca.

%%É válido informar também, que como não há uma equipe dedicada e especializada, o tempo para colaborar com o template é curto. Desta forma, pode ser que não sejam empregadas as melhores técnicas, métodos e ferramentas para o desenvolvimento do template. Também pode acontecer do template não atender completamente todos os anseios e exigências da ABNT e da biblioteca, pois por exemplo, muitas regras de redação possuem questões interpretativas. Assim, o template sempre estará em contínua evolução e seria extremamente interessante que as pessoas (alunos,  professores,  técnicos e entusiastas) colaborarem com a evolução do template. Toda ajuda será bem vinda! Isso pode ser feito enviando e-mail para os desenvolvedores, desta forma, assim que possível esses vão tentar melhorar o template.

%%O template é apenas mais uma ferramenta para o desenvolvimento de trabalhos para a biblioteca. Todavia, podem existir outros templates LaTeX. Assim como há templates em outros formatos, que não o LaTeX. O mais importante é que qualquer pessoa, utilizando a princípio qualquer ferramenta, pode desenvolver textos que atendem os requisitos da biblioteca apenas estudando, interpretando e seguindo as regras da UTFPR e da ABNT, que estão disponíveis na página Web da instituição. O template é só um facilitador.

%%Por fim,  é necessário entender que infelizmente o ambiente LaTeX pode ser complexo e gerar resultados distintos dependendo do: sistema operacional,  pacotes LaTeX utilizados,  configurações alteradas, editor utilizado, a forma que está sendo redigida textos, figuras,  etc. Assim não há como garantir que o resultado final será o esperado.  Dito tudo isso,  >>UTILIZE ESSE TEMPLATE POR SUA CONTA E RISCO<<. Os desenvolvedores e colaboradores deste template não se responsabilizam pelo resultado do uso deste template e se eximem de qualquer responsabilidade.

%###################################################

%% Pacotes carregados nas classes:
%%   memoir: abstract, appendix, array, booktabs, ccaption, chngcntr, chngpage, dcolumn, delarray, enumerate, epigraph, framed,
%%           ifmtarg, ifpdf, index, makeidx, moreverb, needspace, newfile, nextpage, parskip, patchcmd, setspace, shortvrb, showidx,
%%           tabularx, titleref, titling, tocbibind, tocloft, verbatim, verse.
%%   memoir (similares): crop, fancyhdr, geometry, sidecap, subfigure, titlesec.
%%   abntex2: babel, bookmark, calc, enumitem, ifthen, hyperref, textcase.
%%   utfprpgtex: abntex2cite, ae, algorithmic, amsmath, backref, breakurl, caption, cmap, color, eepic, epic, epsfig, etoolbox,
%%               fancyhdr, fix-cm, fontenc, glossaries, graphics, graphicx, helvet, hyphenat, indentfirst, inputenc, lastpage,
%%               morewrites, nomencl, sfmath, sistyle, substr, times, xtab.


%% Pacotes adicionais (\usepackage[options]{package})
\usepackage{bigdelim, booktabs, colortbl, longtable, multirow}%% Ferramentas para tabelas
\usepackage{amssymb, amstext, amsthm, icomma}%% Ferramentas para linguagem matemática
\usepackage{pifont, textcomp, wasysym}%% Símbolos de texto
\usepackage{lipsum}				% para geração de dummy text
\usepackage{subfig}             % para adicionar figuras lado a lado no texto                    
\usepackage{pdfpages}           % para adicionar documentos pdf ao trabalho
\usepackage{xspace}
\usepackage{tocloft}


% luiz: primeira letra maiúscula
% solução 1
%\usepackage{stringstrings}
%\newcommand{\firstcap}[1]{\caselower[e]{#1}\capitalize{\thestring}}

% solução 2 - não usei essa
% \usepackage[utf8]{inputenc}
% \usepackage{datatool-base}
% \usepackage{mfirstuc}

% Formatação do título da seção - primeira letra caixa alta e o resto em caixa baixa.
% \usepackage[explicit]{titlesec}
% \usepackage{lipsum}
% \titleformat{\section}{\normalfont}{\thesection}{1em}{\textbf{\firstcap{#1}}} % funciona mas apenas para o título da seção e não para o sumário (a configuração do sumário está mais para baixo

% luiz: define o underline colorido.
% https://github.com/abntex/abntex2-contrib/blob/master/customizacoes/pucminas/abntex2-pucminas.sty
% acabei não usando o black e o coloruline da solução do link

\usepackage[normalem]{ulem} % para o underline colorido na seção quaternária
\renewcommand*{\cftsubsubsectionfont}{\normalfont\uline} % underline no sumário
\setsubsubsecheadstyle{\ABNTEXsubsubsectionfont\ABNTEXsubsubsectionfontsize\ABNTEXsubsubsectionupperifneeded\uline} %underline no título da subsubsection

% luiz: bibliografia - opções

%% Comandos personalizados (\newcommand{name}[num]{definition})
\newcommand{\cpp}{\texttt{C$++$}}%% C++
\newcommand{\latex}{\LaTeX\xspace}%% LaTeX
\newcommand{\ds}{\displaystyle}%% Tamanho normal das equações
\newcommand{\bsym}[1]{\boldsymbol{#1}}%% Texto no modo matemático em negrito
\newcommand{\mr}[1]{\mathrm{#1}}%% Texto no modo matemático normal (não itálico)
\newcommand{\der}{\mr{d}}%% Operador diferencial
\newcommand{\deri}[2]{\frac{\der #1}{\der #2}}%% Derivada ordinária
\newcommand{\derip}[2]{\frac{\partial #1}{\partial #2}}%% Derivada parcial
\newcommand{\pare}[1]{\left( #1 \right)}%% Parênteses
\newcommand{\colc}[1]{\left[ #1 \right]}%% Colchetes
\newcommand{\chav}[1]{\left\lbrace #1 \right\rbrace}%% Chaves
\newcommand{\sen}{\operatorname{sen}}%% Operador seno
\newcommand{\senh}{\operatorname{senh}}%% Operador seno hiperbólico
\newcommand{\tg}{\operatorname{tg}}%% Operador tangente
\newcommand{\tgh}{\operatorname{tgh}}%% Operador tangente hiperbólico
\newcommand{\seqref}[1]{Equação~\eqref{#1}}%% Referência de uma única equação
\newcommand{\meqref}[1]{Equações~\eqref{#1}}%% Referência de múltiplas equações
\newcommand{\citep}[1]{\cite{#1}}%% Atalho para citação implícita
\newcommand{\citet}[1]{\citeonline{#1}}%% Atalho para citação explícita
\newcommand{\citepa}[1]{(\citeauthor{#1})}%% Atalho para citação implícita (somente autor)
\newcommand{\citeta}[1]{\citeauthoronline{#1}}%% Atalho para citação explícita (somente autor)
\newcommand{\citepy}[1]{(\citeyear{#1})}%% Atalho para citação implícita (somente ano)
\newcommand{\citety}[1]{\citeyear{#1}}%% Atalho para citação explícita (somente ano)

\newcommand{\fonteTexto}[1]{\renewcommand{\familydefault}{#1}}

% Define o caminho das figuras
\graphicspath{{figuras/}}

% Define a fonte ara helvet que é uma fonte similar à Arial, se for usar a Arial tem que mudar o compilador para XeLaTex, mas ai tem que arrumar os erros: https://latex.org/forum/viewtopic.php?t=25998
%\usepackage{helvet}
%\renewcommand{\familydefault}{\sfdefault}
%\usepackage{times} % para fonte time new roman
%\usepackage{pslatex} % ou essa aqui...

%\usepackage{titlesec}

%% Configuração de glossário
% \usepackage[portuguese]{nomencl}
% \usepackage[nogroupskip,nonumberlist,nopostdot,nohypertypes={acronym}]{glossaries}
% \makenoidxglossaries
\usepackage{glossaries}
\makeglossaries

% para siglas em português
\newcommand{\siglaPt}[2]
{
 \newglossaryentry{#1}{
  name=#1,
  description={#2},
  first={#2 (#1)},
  long={#2}
 }  
}

% para siglas de língua estrangeira, nessas a descrição longa fica em itálico.
\newcommand{\siglaIt}[2]
{
 \newglossaryentry{#1}{
  name=#1,
  description={\textit{#2}},
  first={\textit{#2} ({#1})},
  long={\textit{#2}}
 }  
}

%% luiz - para fazer os avisos
\usepackage{tcolorbox}

% use para criar caixas de avisos, pode ser utilizado para fazer anotações de tarefas indicadas pelo orientador/banca.
% \caixa{Atenção}{texto...}
\newcommand{\caixa}[2]{
\begin{tcolorbox}[colback=red!5!white,colframe=red!45!white, title = #1, fonttitle=\bfseries]
#2
\end{tcolorbox}
}

% Luiz - Linhas órfãs e viúvas
\widowpenalty=10000
\clubpenalty=10000

% Luiz - Caption do tamanho da Tabela
%\usepackage[width=1\textwidth]{caption}

% Luiz - configurar a margem dos itens
\setlength{\leftmargini}{1.5cm}
\setlength{\leftmarginii}{1.5cm}
