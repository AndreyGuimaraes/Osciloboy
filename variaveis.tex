%%%% variaveis.tex, 2022/05/02, 2.4a
%%%% Copyright (C) 2020 Vinicius Pegorini (vinicius@utfpr.edu.br)
%%
%% This work may be distributed and/or modified under the conditions of the
%% LaTeX Project Public License, either version 1.3 of this license or (at your
%% option) any later version.
%% The latest version of this license is in
%%   http://www.latex-project.org/lppl.txt
%% and version 1.3 or later is part of all distributions of LaTeX version
%% 2005/12/01 or later.
%%
%% This work has the LPPL maintenance status `maintained'.
%%
%% The Current Maintainer of this work is Vinicius Pegorini.
%% Updated by:
%% - Marco Aurélio Graciotto Silva;
%% - Rogério Aparecido Gonçalves;
%% - Luiz Arthur Feitosa dos Santos.
%%
%% This work consists of the files utfpr.cls, main.tex, and
%% variaveis.tex.
%%
%% A brief description of this work is in readme.txt.

%% ####################################################
%%
%% >> Atenção - Leia isso antes de usar esse template<< 
%%
%% Esse template foi desenvolvido por professores,  com a intenção de ajudar os alunos com as entregas na biblioteca. Não há uma equipe especializada e dedicada mantendo tal template, mas sim professores trabalhando além das suas funções básicas, que são: ensino, pesquisa e extensão.
%
%% Também os mantenedores deste template não são especializados em LaTeX, muito menos em normas da ABNT. Todos que contribuíram com o template fizeram isso visando deixá-lo o mais próximo possível das normas da ABNT e das regras, anseios e expectativas da biblioteca da UTFPR. É muito importante entender que os desenvolvedores do template não têm relação direta com a biblioteca ou com a ABNT. Ou seja, não são os desenvolvedores do template que ditam as regras e normas dos textos que devem ser entregues à biblioteca.

%%É válido informar também, que como não há uma equipe dedicada e especializada, o tempo para colaborar com o template é curto. Desta forma, pode ser que não sejam empregadas as melhores técnicas, métodos e ferramentas para o desenvolvimento do template. Também pode acontecer do template não atender completamente todos os anseios e exigências da ABNT e da biblioteca, pois por exemplo, muitas regras de redação possuem questões interpretativas. Assim, o template sempre estará em contínua evolução e seria extremamente interessante que as pessoas (alunos,  professores,  técnicos e entusiastas) colaborarem com a evolução do template. Toda ajuda será bem vinda! Isso pode ser feito enviando e-mail para os desenvolvedores, desta forma, assim que possível esses vão tentar melhorar o template.

%%O template é apenas mais uma ferramenta para o desenvolvimento de trabalhos para a biblioteca. Todavia, podem existir outros templates LaTeX. Assim como há templates em outros formatos, que não o LaTeX. O mais importante é que qualquer pessoa, utilizando a princípio qualquer ferramenta, pode desenvolver textos que atendem os requisitos da biblioteca apenas estudando, interpretando e seguindo as regras da UTFPR e da ABNT, que estão disponíveis na página Web da instituição. O template é só um facilitador.

%%Por fim,  é necessário entender que infelizmente o ambiente LaTeX pode ser complexo e gerar resultados distintos dependendo do: sistema operacional,  pacotes LaTeX utilizados,  configurações alteradas, editor utilizado, a forma que está sendo redigida textos, figuras,  etc. Assim não há como garantir que o resultado final será o esperado.  Dito tudo isso,  >>UTILIZE ESSE TEMPLATE POR SUA CONTA E RISCO<<. Os desenvolvedores e colaboradores deste template não se responsabilizam pelo resultado do uso deste template e se eximem de qualquer responsabilidade.

%###################################################

%% Documento
%% Luiz: Define a fonte do texto da monografia
\fonteTexto{\sfdefault} % utilize \rmdefault para Times New Roman ou \sfdefault para Arial
\TipoDeDocumento{Trabalho de Conclusão de Curso de Graduação}%% Tipo de documento: "Tese", "Dissertação" ou "Trabalho de Conclusão de Curso de Graduação", "Estágio Supervisionado"
\NivelDeFormacao{Bacharelado}%% Nível de formação: "Doutorado", "Mestrado", "Bacharelado" ou "Tecnólogo" - ATENÇÃO, isso será utilizado para alterar a formatação do trabalho, pois pode haver formatações distintas dependendo o nível/tipo de trabalho.


%% luiz
% Template LaTex criado pelo Departamento Acadêmico de Computação (DACOM)
% da Universidade Tecnológica Federal do Paraná - Campus Campo Mourão (UTFPR-CM)
% Criado e alterado pelos professores:
% - Marco Aurélio Graciotto Silva
% - Rogério Aparecido Gonçalvez
% - Luiz Arthur Feitosa dos Santos
% Esse template utiliza a licença CC BY:
% Esta licença permite que outros distribuam, remixem, adaptem e criem a partir deste trabalho, mesmo para fins comerciais, desde que atribuam o devido crédito pela criação original.
% https://creativecommons.org/licenses/by/4.0/deed.pt_BR

% Dados do curso. Caso seja BCC:
\program{Curso de Bacharelado em Engenharia Elétrica}
\programen{Bachelor's Degree Course in Electrical Engineering}
\degree{Bacharel}
\degreearea{Engenharia Elétrica}
% Caso seja TSI:
% \program{Curso Superior de Tecnologia em Sistemas para Internet}
% \programen{Undergradute Program in Tecnology for Internet Systems}
% \degree{Tecnólogo}
% \degreearea{Tecnologia em Sistemas para Internet}


% Dados da disciplina. Escolha uma das opções e a descomente:
% TCC1:
%\goal{Trabalho de Conclusão de Curso de Graduação}
%\course{Trabalho de Conclusão de Curso 1}
% TCC2:
\goal{Trabalho de Conclusão de Curso de Graduação}
\course{Trabalho de Conclusão de Curso 2}


% Dados do TCC (precisa alterar)
\author{Andrey Alexandre Guimarães\\Rafael Felipe Parolin}  % Seu nome
\authorbib{Guimarães, Andrey A.} % Seu nome para referência bibliográfica (Sobrenome, Nome)
\authorbib{Parolin, Rafael F.}
% Seu nome
% Seu nome para referência bibliográfica (Sobrenome, Nome)
\title{Proposta de um Multimedidor Modular de Baixo Custo com Osciloscópio e Comunicação Sem Fio para Apoio aos Laboratórios da Universidade Tecnológica Federal do Paraná} % Título do trabalho
\titleen{Proposal for a Modular Low-Cost Multimeter with Oscilloscope and Wireless Communication to Support the Laboratories of the Federal University of Technology -- Paraná} % Título traduzido para inglês
\advisor{Prof. Dr. Juan Camilo Castellanos Rodriguez} % Nome do orientador. Lembre-se de prefixar com "Prof. Dr.", "Profª. Drª.", "Prof. Me." ou "Profª. Me."}
% Se não houver coorientador, comente a linha a baixo
\coadvisor{Prof. Dr. Clóvis Ronaldo da Costa Bento} % Nome do coorientador, caso exista. Caso não exista, comente a linha.
\depositshortdate{2024} % Ano em que depositou este documento
\approvaldate{xx/Outubro/2024}

% Dados do curso que não precisam de alteração
\university{Universidade Tecnológica Federal do Paraná (UTFPR)}
\universityen{Federal University of Technology -- Paraná}
\universitycampus{Campus Curitiba}
\universityunit{Departamento Acadêmico de Eletrotécnica}
\address{CURITIBA}
\addressen{Curitiba, PR, Brazil}
\documenttype{Monografia}
\documenttypeen{Monograph}
\degreetype{Graduação}

\evalboardmember{Juan Camilo Castellanos Rodriguez}{Doutorado}{Universidade Tecnológica Federal do Paraná}
\evalboardmember{Nastasha Salame da Silva}{Doutorada}{Universidade Tecnológica Federal do Paraná}
\evalboardmember{Clóvis Ronaldo da Costa Bento}{Doutorado}{Universidade Tecnológica Federal do Paraná}
% \evalboardmember{Roberto Zanetti Freire}{Dr.}{Universidade Tecnológica Federal do Paraná}

% \evalboardmember{Nome completo e por extenso do Membro 4}{Título (especialização, mestrado, doutorado}{Nome completo e por extenso da instituição a qual possui vínculo}

%% Palavras-chave e keywords
%% ATENÇÃO - você deve indicar a quantidade de palavras chaves para o template LaTeX utilizar o pontuação correta!
\NumeroDePalavrasChave{3}%% Número de palavras-chave (máximo 5)
%% Atenção - por enquanto o template não está suportando acentos normais na palavra chave, por isso caso a palavra tenha acento, você deve utilizar o estilo antigo do LaTeX, sendo os acentos: á - \'a  é - \'e   â - \^a  ê - \^e  à - \`a  ä - \"a  ç - \c{c}
\PalavraChaveA{multimedidor com oscilosc\'opio}%% Palavra-chave A
\PalavraChaveB{comunica\c{c}\~ao sem fio}%% Palavra-chave B
\PalavraChaveC{baixo custo}%% Palavra-chave C
% \PalavraChaveD{baixo-custo}%% Palavra-chave D
% \PalavraChaveE{ondas}%% Palavra-chave E
%% Exemplo de como utilizar acentos na Palavra-chave:
% \PalavraChaveA{ol\'a}%% Olá
%\PalavraChaveB{voc\^e}%% você
%\PalavraChaveC{\`a}%% à
%\PalavraChaveD{a\c{c}\~ao}%% ação
%\PalavraChaveE{arg\"uir}%% argüir


%% ATENÇÃO - você deve indicar a quantidade de keywords para o template LaTeX utilizar o pontuação correta!
\NumeroDeKeywords{3}%% Número de keywords (máximo 5)
\KeywordA{multimeter with oscilloscope}%% Keyword A
\KeywordB{wireless communication}%% Keyword B
\KeywordC{low cost}%% Keyword C
% \KeywordD{low-cost}%% Keyword D
% \KeywordE{waves}%% Keyword E


% É obrigatório o uso de uma licença Creative Commons (CC) nos trabalhos de TCC pelos cursos ligados a DACOM da UTFPR-CM.
% Veja: http://portal.utfpr.edu.br/biblioteca/trabalhos-academicos/docentes/procedimento-de-entrega-graduacao

% Sendo assim, escolha com o seu orientador uma das licenças CC a seguir: 

% CC BY: Esta licença permite que outros distribuam, remixem, adaptem e criem a partir deste trabalho, mesmo para fins comerciais, desde que atribuam o devido crédito pela criação original. Essa é a menos restritiva.
\licenca{ccby}

% CC BY CA: Esta licença permite que outros remixem, adaptem e criem a partir deste trabalho, mesmo para fins comerciais, desde que atribuam o devido crédito e que licenciem as novas criações sob termos idênticos.
%\licenca{ccbysa}

% CC BY ND: Esta licença permite a redistribuição, comercial e não comercial, desde que o trabalho seja distribuído inalterado e no seu todo, com crédito ao autor.
%\licenca{ccbynd}

% CC BY NC: Esta licença permite que outros remixem, adaptem e criem a partir deste trabalho para fins não comerciais, e embora os novos trabalhos tenham de atribuir o devido crédito e não possam ser usados para fins comerciais, os trabalhos derivados não têm que serem licenciados sob os mesmos termos.
%\licenca{ccbync}

% CC BY NC SA: Esta licença permite que outros remixem, adaptem e criem a partir deste trabalho para fins não comerciais, desde que atribuam ao autor o devido crédito e que licenciem as novas criações sob termos idênticos.
%\licenca{ccbyncsa}

% CC BY NC ND: Esta licença só permite que outros façam download do trabalho e o compartilhe desde que atribuam crédito ao autor, mas sem que possam alterá-los de nenhuma forma ou utilizá-los para fins comerciais. Essa é a mais restritiva.
%\licenca{ccbyncnd}

% Deixar sem licença - isso é aplicado apenas aos trabalhos que não são obrigados a ter licença. Na duvida verifique isso com o seu orientador e professor responsável pelo TCC. Para deixar o texto sem licença deixe o comando licença em brando ou deixe comentado.
%\licenca{}
% by DACOM/UTFPR-CM